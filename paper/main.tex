\documentclass[fleqn,10pt]{wlscirep}
\usepackage[utf8]{inputenc}
\usepackage[T1]{fontenc}
\usepackage{graphicx}
\usepackage{siunitx}
\usepackage{subcaption}
\usepackage{xargs}  % Use more than one optional parameter in a new commands
\usepackage[colorinlistoftodos,prependcaption]{todonotes}

\graphicspath{{figures/}}
\sisetup{locale=US,group-minimum-digits=5,group-separator={,}}

\newcommandx{\alltodo}[2][1=]{\todo[author=Everyone,linecolor=red,backgroundcolor=red!25,bordercolor=red,#1]{#2}}
\newcommandx{\mctodo}[2][1=]{\todo[author=Matt,linecolor=blue,backgroundcolor=blue!25,bordercolor=blue,#1]{#2}}
\newcommandx{\rokemtodo}[2][1=]{\todo[author=Ariel,linecolor=green,backgroundcolor=green!25,bordercolor=green,#1]{#2}}
\newcommandx{\comment}[2][1=]{\todo[author=Comment,linecolor=lime,backgroundcolor=lime!25,bordercolor=lime,#1]{#2}}
\newcommandx{\arhtodo}[2][1=]{\todo[author=Adam,linecolor=purple,backgroundcolor=purple!25,bordercolor=purple,#1]{#2}}

\title{A preprocessed open diffusion derivatives dataset from the Healthy Brain Network}

\author[1,*$\dagger$]{Adam Richie-Halford}
\author[2,$\dagger$]{Matthew Cieslak}
\author[4]{Lei Ai}
\author[5]{Sendy Caffarra}
\author[4]{Alexandre R. Franco}
\author[5]{Iliana Karipidis}
\author[3]{John Kruper}
\author[4]{Michael Milham}
\author[5]{Barbara Avelar Pereira}
\author[5]{Ethan Roy}
\author[2]{Valerie J. Sydnor}
\author[5]{Jason Yeatman}
\author[6]{The Fibr Community Science Consortium}
% \author[6]{Jul\_dugre}
\author[6]{amandamcgow}
\author[6]{SimonMB}
\author[6]{Nabbott}
\author[6]{AmjadSamara}
\author[6]{Shannon Grogans}
\author[6]{Zzzdeng}
\author[6]{HamidTurker}
\author[6]{Dragon}
\author[6]{Lshaffe6}
\author[6]{Hilarym}
\author[6]{faskybrain}
\author[6]{RaeYuan}
\author[6]{wanderinggrace}
\author[6]{sofie}
\author[6]{llotter}
\author[6]{valeriejill}
\author[6]{AkiNikolaidis}
\author[6]{Deagaric}
\author[6]{akoz}
\author[6]{summerf}
\author[6]{lindenmp}
\author[6]{mjl0062}
\author[6]{Nevena}
\author[6]{kat\_sh}
\author[6]{LeonF}
\author[6]{anjaue}
\author[6]{utooley1}
\author[6]{adrian\_neurosci}
\author[6]{mattwall}
\author[6]{Griffin SH}
\author[6]{jbourque}
\author[6]{averigiudicessi}
\author[6]{ncolenbi}
\author[6]{vtrip}
\author[6]{MeganHimes}
\author[6]{ti}
\author[6]{floundersmw}
\author[6]{itsyuchao}
\author[6]{jjfc}
\author[6]{Agwiesman}
\author[6]{bhaggerty}
\author[6]{benjamindeleener}
\author[6]{AnthonyJa}
\author[6]{seedyh}
\author[6]{nm\_fibr}
\author[6]{matthew2}
\author[6]{Li}
\author[6]{Gabriel Gonzalez-Escamilla}
\author[6]{Gianpaolo}
\author[6]{Mareike}
\author[6]{arielleered}
\author[6]{gdelap}
\author[6]{elliem}
\author[6]{skorenic}
\author[6]{Mark Wiesman}
\author[6]{SHKohl}
\author[6]{claudiap}
\author[6]{mackenzie}
\author[6]{jaewsong}
\author[6]{guillaume}
\author[6]{sadieneuro}
\author[6]{GuiomarNiso}
\author[6]{PSB}
\author[6]{elorenc}
\author[6]{smarek}
\author[6]{fsayed}
\author[6]{gthf61448}
\author[6]{WendyWang}
\author[6]{dcg2153}
\author[6]{mpeter55}
\author[6]{cmich}
\author[6]{ikahhale}
\author[6]{czaj}
\author[6]{JodyFinch11}
\author[6]{albie}
\author[6]{holoships}
\author[6]{ckw84}
\author[6]{Gaga}
\author[6]{ecoffey}
\author[6]{jmayor}
\author[6]{jbryn}
\author[6]{jlurp20}
\author[6]{ShaoshiZ}
\author[6]{slhkam}
\author[6]{ckorponay}
\author[6]{BrunoHeblingVieira}
\author[6]{Yacine Mahdid}
\author[6]{harriet89}
\author[6]{Christian}
\author[6]{JonathanTSchneider}
\author[6]{bennewman}
\author[6]{sidchop}
\author[6]{Smeisler}
\author[6]{roryboyle}
\author[6]{cMadan}
\author[6]{MaryLena}
\author[6]{mkp}
\author[6]{Twelton}
\author[6]{dalez}
\author[6]{hyruuk}
\author[6]{ehhall}
\author[6]{jlhanson5}
\author[6]{smomara}
\author[6]{EvrenOzarslan}
\author[6]{ctw}
\author[6]{mollyr}
\author[6]{NChaku}
\author[6]{iliazaharov}
\author[6]{askeller}
\author[6]{clh64}
\author[6]{flat\_brainer}
\author[6]{H Farmer}
\author[6]{Jacobjcrouse}
\author[6]{Ryguy}
\author[6]{VincentBottom}
\author[6]{aomary}
\author[6]{aojha7}
\author[6]{wanderine}
\author[6]{Kahini}
\author[6]{bc1115}
\author[6]{adam}
\author[6]{sasims}
\author[6]{anovan}
\author[6]{patriciaschm}
\author[6]{dlsta}
\author[6]{berwe48}
\author[6]{Token Tech Chick}
\author[6]{Michael Cormican}
\author[6]{mhettwer}
\author[6]{pkirk}
\author[6]{lexlosia}
\author[6]{n00bmaster69}
\author[6]{Greki}
\author[6]{GuidoG}
\author[6]{ben}
\author[6]{yhahn1819}
\author[6]{GiuliaLiberati}
\author[6]{skyesinc}
\author[6]{djdlc85}
\author[6]{janderz8}
\author[6]{aiwiesman}
\author[6]{kurtzoner}
\author[6]{Dwin}
\author[6]{harman}
\author[2,$\ddagger$]{Theodore D. Satterthwaite}
\author[3,1,$\ddagger$]{Ariel Rokem}

\affil[1]{University of Washington, eScience Institute, Seattle, Washington, 98195, USA}
\affil[2]{University of Pennsylvania, Department of Psychiatry, Philadelphia, Pennsylvania, 19104, USA}
\affil[3]{University of Washington, Department of Psychology, Seattle, Washington, 98195, USA}
\affil[4]{Child Mind Institute, New York City, 10022, USA}
\affil[5]{Stanford University, Graduate School of Education and Division of Developmental and Behavioral Pediatrics, Stanford, California, 94305, USA}
\affil[6]{The Fibr Community Science Consortium}

\affil[*]{richford@uw.edu}
\affil[$\dagger$]{these authors contributed equally to this work}
\affil[$\ddagger$]{these authors contributed equally to this work}

%\keywords{Keyword1, Keyword2, Keyword3}

\begin{abstract}
\arhtodo[inline]{Write abstract}
\rokemtodo[inline]{Any input on the abstract is much appreciated.}
\end{abstract}
\begin{document}

\flushbottom
\maketitle
\thispagestyle{empty}

\todo[inline]{%
    Note to editing authors:

    We use the \texttt{todonotes} package to keep track of remaining tasks and
    comments. You can add a task for Adam with the
    \texttt{\textbackslash{}arhtodo} command, a task for Matt with the
    \texttt{\textbackslash{}mctodo} command, a task for Ariel with the
    \texttt{\textbackslash{}rokemtodo} command, a task for all reviewing authors
    with the \texttt{\textbackslash{}alltodo} command, and a general comment
    with the \texttt{\textbackslash{}comment} command.
}
\comment[inline]{Comments look like this.}
\arhtodo[inline]{Tasks for Adam look like this.}
\rokemtodo[inline]{Tasks for Ariel look like this.}
\mctodo[inline]{Tasks for Matt look like this.}
\alltodo[inline]{Tasks for all reviewing authors look like this.}

\section*{Introduction}

Childhood and adolscence are a time of rapid dynamic change to human brain
function. It is also a time during which susceptibility to mental health
disorders tends to manifest. Understanding individual differences in brain
development and understanding the interplay between brain development and the
development of these disorders requires measurements from many invidividuals.
\arhtodo[inline]{%
    cite studies/reviews that show the need for large sample sizes;
    citations for other statements here also needed
}

The Healthy Brain Network (HBN) is a landmark pediatric mental health study
collecting MRI images and clinical assessment data from \num{10000} New York
City area children and adolescents \cite{alexander2017-yc}. The HBN dataset
contains a wealth of phenotypic and imaging data, including diffusion MRI (dMRI)
data, which allows for analysis of the physical properties of developing white
matter \cite{wandell2016-qt}. This dMRI data is openly available in raw form
through the Functional Connectomes Project and the International Neuroimaging
Data-Sharing Initiative (FCP-INDI), spurring collaboration on open big-data
reproducible science \cite{avesani2019-ey}. However, even with the data publicly
available, several steps need to be completed before the data can be fruitfully
analyzed.

First, analysis of dMRI data must start with a pipeline of critical
preprocessing steps, such as eddy current correction, motion correction, and
adjustment of the gradient directions. Because of the complexity of some of
these steps, investigators may neglect to perform some preprocessing steps or
may make errors that can induce bias in their subsequent interpretation of the
data \cite{jones2010-ps}. Furthermore, once preprocessing is done correctly and
transparently once, there is little need for researchers to repeat this step.
Thus, there is a need for an openly available preprocessed diffusion derivative
dataset that applies best practices in preprocessing in a robust and transparent
way \cite{cieslak2021-iq}. Accordingly, here we introduce the HBN Preprocessed
Open Diffusion Derivatives (HBN-POD2), a large dataset for the analysis of
structural brain connectivity and pediatric mental health.

Second, dMRI measurements are susceptible to a variety of artifacts that affect
the quality of the signals and the ability make accurate inferences from them.
In small studies, with few participants, it is common to thoroughly examine the
data from every participant as part of a quality control (QC) process. Expert
examination is time consuming and is prohibitive in large datasets, such as HBN.
This difficulty could be ameliorated through the automation of QC. Thanks to
their success in many other visual recognition tasks previously only achievable
by humans, machine learning and computer vision methods, such as convolutional
deep artificial neural networks, or ``deep learning'' \cite{lecun2015deep}
are promising avenues for automation of QC. One
of the challenges of these new methods is that they require a large training
dataset to attain accurate performance. In previous work, we demonstrated that
deep learning can accurately emulate expert examination of T1-weighted brain
images for QC \cite{keshavan2019-er}. To obtain a large enough training dataset of
T1w images in that case, we deployed a community science application that
collected quality control scores of parts of the dataset from volunteers through
a web application \footnote{%
    While the legacy term ``citizen science'' evokes a sense of civic duty in
    scientific engagement, it can also imply a barrier for community members who
    want to contribute to science but may not be citizens of a particular
    country.  In this manuscript we use the more modern term ``community
    science.''
}.
These scores were then calibrated using a gold standard expert-scored subset of
these images. A deep learning neural network was trained on the calibrated and
aggregated score, resulting in very high concordance with expert ratings on a
separate test dataset. We term this approach ``hybrid QC'', because it combines
information from experts with information from community scientists to create a
scalable machine learning algorithm that can be applied to future data
collection.

However, the hybrid QC proof-of-concept left lingering questions about its
applicability to other datasets because it was trained on a single-site,
single-modality dataset.  Here, we expand the hybrid-QC approach to a large
multi-site diffusion MRI dataset. This addressed the generalizability concerns
of the hybrid QC approach while also providing a large openly available
pediatric diffusion MRI dataset.  At the same time, we provide preliminary
evidence that the deep learning model is learning relevant QC features from the
diffusion weighted images.  One of the common critiques of deep learning is that
it can learn irrelevant features of the data and does not provide information
that is transparent enough to interpret \cite{lipton2017doctor,
salahuddin2022transparency}.
\arhtodo[inline]{%
    There's probably more to cite here. e.g. that famous example where the
    computer vision models were focusing on clinical annotations in x-rays
    rather than the images themselves. Where is that paper?
}
Therefore, we used machine learning interpretation methods in order to confirm
whether the deep learning algorithm uses meaningful features of the images to
perform accurate QC \cite{sundararajan2017axiomatic,murdoch2019definitions}.

Taken together, BIDS-curated data, processed data and quality control scores
generated by the deep learning algorithm provide researchers with everything
that they need in order to start applying a large variety of analysis approaches
to HBN-POD2 and to help answer the important questions that this data was
collected to address.

\arhtodo[inline]{%
    Intro needs a lot more meat with relevant lit on importance of
    open datasets, dMRI, and QC.
}

\section*{Results}

The aims of this study were fourfold
\begin{enumerate*}[%
    label=(\roman*),%
    before=\unskip{: },%
    itemjoin={{, }},%
    itemjoin*={{, and }}]
    \item curate the HBN MRI data into a fully-BIDS compliant MRI dataset
    \item perform state-of-the-art diffusion preprocessing using \emph{QSIPrep}
    \item assign QC scores to each subject
    \item provide unrestricted public release of the outputs from each of these
    steps.
\end{enumerate*}

Figure~\ref{fig:hbn-sankey} depicts the data provenance of the HBN-POD2 dataset.
Starting with MRI data from \num{2747} HBN participants available on FCP-INDI,
we curated these data for compliance with the Brain Imaging Data Structure
(BIDS) specification \cite{gorgolewski2016-lh} and preprocessed the structural
MRI (sMRI) and diffusion MRI (dMRI) data using \emph{QSIPrep}, resulting in
\num{2136} subjects with preprocessed, BIDS-compliant dMRI data.

\begin{figure}[ht]
    \centering
    \includegraphics[width=0.75\linewidth]{hbn-pod2-sankey.png}
    \caption{%
        {\bf HBN-POD2 data provenance}:
        Imaging data for \num{2747} subjects, aged 5-21 years and collected at four
        sites in the New York City area, was made available through the
        Functional Connectomes Project and the International Neuroimaging
        Data-Sharing Initiative (FCP-INDI).
        %
        These data were curated for compliance to the BIDS specification
        \cite{gorgolewski2016-lh} and availability of imaging metadata in json
        format. \num{2615} subjects met this specification.
        %
        Imaging data was preprocessed using \emph{QSIPrep} \cite{cieslak2021-iq}
        to group, distortion correct, motion correct, denoise, coregister and
        resample MRI scans. Of the BIDS curated subjects, \num{2136} subjects
        passed this step, with the majority of failures coming from subjects
        with missing dMRI scans.
        %
        Expert raters assigned QC scores to \num{200} of these participants,
        creating a ``gold standard'' QC subset. Community raters then assigned
        binary QC ratings to a superset of the gold standard containing
        \num{1653} participants. An image classification algorithm was trained
        on a combination of automated QC metrics from QSIPrep and community
        scientist reviews to ``extend'' the expert ratings to the community
        science subset.  Finally, a deep learning QC model was trained on the
        community science subset to assign QC scores to the entire dataset and
        to future releases from HBN.
        %
        The HBN-POD2 dataset, including QC ratings, is openly available through
        FCP-INDI.
    }
    \label{fig:hbn-sankey}
\end{figure}

To achieve the QC aim of the project, we adopted a hybrid QC approach that
combines expert rating, community science, and deep learning, drawing on the success of a previous application in
assessing the quality of HBN's structural MRI data \cite{keshavan2019-er}.
This method
\begin{enumerate*}[%
    label=(\roman*),%
    before={{ }},%
    itemjoin={{, }},%
    itemjoin*={{ and }}]
    \item starts with dMRI expert raters labelling a small subset of subjects,
    the ``gold standard'' dataset
    \item amplifies these labels using a community science web application to
    extend expert ratings to a much larger subset of the data, the community
    science subset
    \item trains a deep learning model on the community science subset to
    predict expert decisions on the entire dataset.
\end{enumerate*}

\subsection*{Expert quality control}

To create the gold standard dataset, we first developed \emph{dmriprep-viewer},
a dMRI data viewer and QC rating web application to display \emph{QSIPrep}
outputs and collect expert ratings \cite{richie-halford2021-viewer}. Six of the
co-authors, who are all dMRI experts, rated a 200-participant subset of the
HBN-POD2 data using extensive visual examination of each subjects dMRI data,
including the diffusion weighting imaging (DWI) time series itself, a plot of
motion parameters through the DWI scan, and full 3D volumes depicting
\begin{enumerate*}[%
    label=(\roman*),%
    before={{ }},%
    itemjoin={{, }},%
    itemjoin*={{ and }}]
    \item the brain mask and $b=0$ to T1w registration
    \item a diffusion tensor imaging (DTI) tensor fractional anisotropy (FA)
    model visualized over the $b=0$ volume.
\end{enumerate*}
The experts rated subjects using a five-point scale with ratings of ``definitely
fail,'' ``probably fail,'' ``unsure,'' ``probably pass,'' and ``definitely
pass.'' Figure \ref{fig:expert-qc:scatter} shows the distribution of expert
ratings and the relationship between the mean expert ratings and some of the
automated QC metrics output by \emph{QSIPrep}: neighboring DWI correlation (NDC)
\cite{yeh2019-kb}, maximum relative translation, and number of outlier slices.
NDC characterizes the pairwise spatial correlation between pairs of DWI volumes
that sample neighboring points in $q$-space. Since lower values indicate reduced
data quality, it is reassuring that NDC correlates directly with expert rating.
Conversely, high relative translation and a high number of motion outlier slices
reflect poor data quality and so these metrics are inversely related to mean
expert rating.

In addition to agreeing qualitatively with \emph{QSIPrep}'s automated QC
metrics, the expert raters also agreed with each other. To demonstrate
inter-rater reliability (IRR), in Figure \ref{fig:expert-qc:irr} we show the
pairwise Cohen's $\kappa$ \cite{di-eugenio2004-bb} for each pair of expert
raters as well as an additional XGB rater which we describe in the next
subsection.

In addition to the pairwise $\kappa$, we also present the intra-class
correlation (ICC) \cite{hallgren2012-ze} as a measure of IRR.  ICC3 and ICC3k
are appropriate variants of ICC when a fixed set of $k$ raters rate each subject
(i.e. a fully crossed design), which is what we have for HBN-POD2. When all
subjects are coded by multiple raters and the average of their ratings is used
for hypothesis testing, ICC3k is appropriate.  When a subset of subjects is
coded by multiple raters and the reliability of their ratings is meant to
generalize to other subjects rated by only one coder, the single-measure ICC3
must be used. The high ICC3k value for each expert rater entitles us to use the
mean expert rating as training data for the XGB algorithm.  And the high ICC
value after inclusion of the XGB model entitles us to treat XGB as a seventh
rater, thereby extending the expert ratings to the entire community science
subset.

\begin{figure}[ht]
    \begin{subfigure}{.5\textwidth}
    \centering
    \includegraphics[width=\linewidth]{community-qc/expert-qsiprep-pairplot-top.pdf}
    \includegraphics[width=\linewidth]{community-qc/expert-qsiprep-pairplot-bottom.pdf}
    \caption{}
    \label{fig:expert-qc:scatter}
    \end{subfigure}
    \begin{subfigure}{.5\textwidth}
    \centering
    \includegraphics[width=\linewidth]{community-qc/expert-raters-cohens-kappa.pdf}
    \caption{}
    \label{fig:expert-qc:irr}
    \end{subfigure}
    \caption{%
        {\bf Expert QC results}:
        Six dMRI experts rated a subset of \num{200} subjects.
        \textbf{(a)} Experts agree with \emph{QSIPrep}'s automated QC metrics.
        Here we show associations between the expert QC rating and the
        \emph{QSIPrep} metrics neighboring diffusion-weighted imaging
        correlation (NDC) \cite{yeh2019-kb}, maximum relative translation, and
        number of outlier slices. As expected, NDC is directly correlated with expert rating while
        the other two metrics are inversely correlated with expert rating.
        %
        \textbf{(b)} Experts agree with each other. Here we show the pairwise
        Cohen's $\kappa$ measure of inter-rater reliability.  One can also use
        interclass correlation to demonstrate interrater reliability. The expert
        raters achieved an $\textbf{ICC3k} = 0.930 (95\% CI: [0.91, 0.94])$,
        $\textbf{ICC3} = 0.688 (95\% CI: [0.64, 0.74])$, $\langle \kappa \rangle
        = 0.648$.  We trained a gradient boosting model (XGB) to predict expert
        ratings based on community science ratings and automated QC metrics,
        obtaining a cross-validated ROC-AUC of $0.96 \pm 0.01$. Treating
        XGB as a seventh expert rater, preserves inter-rater reliability:
        $\textbf{ICC3k} = 0.945 (95\% CI: [0.93, 0.96])$, $\textbf{ICC3} = 0.709
        (95\% CI: [0.66, 0.75])$.  This entitles us to extend it's ratings to
        the community science dataset.
    }
    \label{fig:expert-qc}
\end{figure}

\subsection*{Community science quality control}

Although the expert raters achieved an acceptable inter-rater reliability and
yielded intuitive associations with \emph{QSIPrep}'s automated QC metrics,
generating expert QC labels for the entire HBN-POD2 dataset would be
prohibitively time consuming. To assess the image quality of the remaining
subjects, we released \emph{Fibr} (\url{https://fibr.dev}), a community science
web application in which users assign binary pass/fail labels assessing the
quality of horizonal slice tensor FA images. That is, \emph{Fibr} users saw
individual slices or an animated sequence of slices taken from the entire tensor
FA volume that the expert raters saw. The \emph{Fibr} users therefore saw only a
subset of the imaging data that the dMRI experts had access to. In total,
\num{374} community scientists provided \num{587778} ratings for a mean of $>50$
ratings per slice (or $>200$ ratings per subject). Of these, \num{145} raters
provided $>3,000$ votes each and are included in the Fibr Community Science
Consortium as co-authors on this paper.

We then trained a gradient boosting model \cite{chen2016-eb} to predict expert
scores based on a combination of community science ratings and automated \emph{QSIPrep}
QC metrics. We refer to this model as XGB.  To clarify the contributions of the
automated QC metrics and the community science raters, we trained two additional
gradient boosting models
\begin{enumerate*}[%
    label=(\roman*),%
    before=\unskip{: },%
    itemjoin={{, }},%
    itemjoin*={{ and }}]
    \item one trained only on the automated \emph{QSIPrep} QC metrics, which we
    call XGB-q
    \item one trained on only the \emph{Fibr} ratings, which we call XGB-f.
\end{enumerate*}
XGB-f may be viewed as a data-driven weighting of community scientists' ratings,
while XGB-q may be viewed as a generalization of QC metric exclusion criteria.
In Figure \ref{fig:fibr-qc:scatter}, we show the associations between the mean
expert rating and both the mean \emph{Fibr} rating and the XGB predictions.  The
unadjusted \emph{Fibr} ratings are overly optimistic; on average, community
scientists are not as critical as the expert raters. The XGB predictions remedy
this by
\begin{enumerate*}[%
    label=(\roman*),%
    before={{ }},%
    itemjoin={{, }},%
    itemjoin*={{ and }}]
    \item incorporating automated QC metrics from \emph{QSIPrep}
    \item weighting the contribution of \emph{Fibr} raters who agree more with
    the experts on the training set.
\end{enumerate*}
Figure \ref{fig:fibr-qc:roc} depicts the ROC curves for the XGB, XGB-q, and
XGB-f models.
XGB attains a cross validated area under the receiver operating curve (ROC-AUC) of
$0.96 \pm 0.01$ on the ``gold standard,'' where the $\pm$ indicates the standard
deviation of scores from repeated $k$-fold cross-validation.
In contrast, XGB-q attained an ROC-AUC of $0.91 \pm 0.03$ and XGB-f achieved an
ROC-AUC of $0.84 \pm 0.04$.
The enhanced performance of XGB-q over XGB-f shows that community scientists
alone are not as good as automated metrics at predicting expert ratings. And
yet, the increased performance of XGB over XGB-q demonstrates that there is
additional image quality information to be gained by visual inspecting dMRI
data.

\begin{figure}[tbp]
    \begin{subfigure}{.55\textwidth}
    \centering
    \includegraphics[width=\linewidth]{community-qc/fibr-rating-scatter-plot.pdf}
    \caption{}
    \label{fig:fibr-qc:scatter}
    \end{subfigure}
    \begin{subfigure}{.45\textwidth}
    \centering
    \includegraphics[width=\linewidth]{community-qc/xgb-roc-curve.pdf}
    \caption{}
    \label{fig:fibr-qc:roc}
    \end{subfigure}
    \caption{%
        {\bf Community science predictions of the expert ratings}:
        \textbf{(a)} Scatterplots showing the relationship between mean expert
        rating and both mean \emph{Fibr} rating (left) and XGB prediction
        (right). \emph{Fibr} raters overestimate the quality of images compared
        to expert raters. But the XGB prediction compensates for this by
        incorporating automated QC metrics and weighting more valuable
        \emph{Fibr} raters.
        %
        \textbf{(b)} ROC curves for the XGB, XGB-q, and XGB-f models.
    }
    \label{fig:fibr-qc}
\end{figure}

\subsection*{Automated QC labelling through deep learning}

The XGB ``rater'' does a good job of extending QC ratings to the entire
community science subset, but requires \emph{Fibr} scores. Without community
science ratings, it would fall back on the suboptimal XGB-q prediction. Instead,
we would like a fully automated QC approach that can be readily applied to new
data releases from HBN. We therefore trained a deep convolution neural network
to predict the XGB ratings from \emph{QSIPrep} outputs alone. We modified an
existing 3D convolutional neural network (CNN) architecture \cite{zunair2020-bs},
previously applied to the ImageCLEF Tuberculosis Severity Assessment 2019
benchmark \cite{dicente2019clef} to accept multichannel input containing the
B=0 reference diffusion image, a tensor fractional isotropy (FA) image, and,
optionally, automated QC metrics from QSIPrep. We trained this network on the
XGB ``rater's'' predictions and validated it against the gold standard dataset.
Figure \ref{fig:dl-qc:roc} depicts the ROC curves for two deep learning models,
one trained with additional automated QC metrics from QSIPrep and one trained
without. We refer to the convolutional neural network model training only on
imaging as CNN-i and the model that incorporates automated QC metrics as
CNN-i+q. The two models perform nearly identically and achieve and ROC-AUC of
$0.947$. The identical performance suggests that QSIPrep's automated QC metrics
provide information that is redundant to the lower-level imaging data.
Both CNN-i and CNN-i+q are outperform the XGB-q, which was trained only on
automated QC metrics. While CNN-i and CNN-i+q underperform the full XGB model,
they do not require the additional \emph{Fibr} scores and the associated
logistical burden of collecting community science ratings.

\begin{figure}[htbp]
    \begin{subfigure}{.25\textwidth}
    \centering
    \includegraphics[width=\linewidth]{deep-learning-qc/dl_roc_auc_curve.pdf}
    \caption{}
    \label{fig:dl-qc:roc}
    \end{subfigure}
    \begin{subfigure}{.25\textwidth}
    \centering
    \includegraphics[width=\linewidth]{bundle-profiles/qc-age-jointplot.pdf}
    \caption{}
    \label{fig:dl-qc:joint}
    \end{subfigure}
    \begin{subfigure}{.5\textwidth}
    \centering
    \includegraphics[width=\linewidth]{bundle-profiles/qc-hist.pdf}
    \caption{}
    \label{fig:dl-qc:hist}
    \end{subfigure}
    \caption{%
        {\bf Deep learning QC scores}:
        \textbf{(a)} ROC curves for two deep learning models: one trained with
        additional automated QC metrics from QSIPrep (blue) and one trained
        without (orange). The models perform roughly identically, indicating the
        the QC metrics provide information that is redundant to the imaging.
        Both outperform the XGB-q predictions, indicating the added value of
        the diffusion weighted imaging. While both models underperform the
        XGB predictions, they do not incur the burden of collecting community
        science ratings.
        %
        \textbf{(b)} Joint distributions showing a strong direct association
        between age and QC score. This likely reflects the well-known negative
        association between age and head motion in pediatric neuroimaging.
        %
        \textbf{(c)}
        Histograms showing the relationship between participants QC scores and
        their sex (left) and scan site (right). QC distributions are independent
        of sex and scanning site
    }
    \label{fig:dl-qc}
\end{figure}

While HBN-POD2 provides three QC ratings: the mean expert QC ratings, the XGB
predicted score, and the CNN-i predicted score, we treat the CNN-i score as
definitive QC score because it is available for all participants and can be
easily calculated for new participants in future HBN releases. When we refer to
a participant's QC score without specifying a generating model, the CNN-i score
is assumed. As such, Figure \ref{fig:dl-qc} depicts the distribution of QC
scores by age (Figure \ref{fig:dl-qc:joint}), and sex and scanning site (Figure
\ref{fig:dl-qc:hist}). QC distributions are similar for each
scan site and for male and female participants \footnote{%
    Responses for the sex variable in HBN phenotypic data are limited to
    ``male'' and ``female.''
}. Unsurprisingly, the QC score is strongly directly correlated with age. This
accords with the negative association between head motion and age, which is well
established both in general
\cite{power2012spurious,satterthwaite2012impact,fair2012distinct,yendiki2014spurious}
and in the HBN dataset \cite{alexander2017-yc}.

\subsection*{Attribution masks for the deep learning classifier}

The predictive performance depicted in Figure \ref{fig:dl-qc:roc} gives us
confidence that the trained CNN-i model can classify unseen data, justifying its
extension to the entire HBN-POD2 dataset and to future releases. However, it
does not explain its decisions. As deep learning models have been increasingly
applied to medical image analysis, there is an evolving interest in the
transparency of these models \cite{salahuddin2022transparency}. While an
exhaustive interpretation of deep learning QC models is beyond the scope of this
work, we provide preliminary qualitative interpretation of the CNN-i model.

We generated post-hoc attribution maps that highlight regions of the input
volume that are relevant for the QC prediction. The integrated gradient method
\cite{sundararajan2017axiomatic} is a gradient-based attribution method
\cite{ancona2019gradient} that aggregates gradients for synthetic images
interpolating between a baseline image and the input image. It has been used to
interpret deep learning models applied to diabetic retinopathy prediction
\cite{sayres2019using} and multiple sclerosis diagnosis
\cite{wargnier-dauchelle2021interpretable}. Our goal is to confirm that the
CNN-i model is looking at the same features that an expert rater would look at,
thereby bolstering the decision to apply it to new data. Figure \ref{fig:ig}
shows the attribution maps for example participants from each confusion class:
true positive, true negative, false positive, and false negative. The columns
correspond to the different channels of the deep learning input volume: the B=0
reference image and the DTI FA in the $x$, $y$, and $z$ directions.  The true
positive map (Figure \ref{fig:ig:true-pos}) indicates that the network is
looking at the entire brain rather than focusing on any one anatomical region.
Moreover, the model identifies white matter fascicles that travel along the
direction of the input channel: lateral for $x$, anterior-posterior for $y$, and
superior-inferior for $z$.
The true negative attribution map (Figure \ref{fig:ig:true-neg}) reveals that
when the reference B=0 volume contains motion artifacts, such as banding, the
network ignores the otherwise positive attributions for the clearly identifiable
white matter tracts in the DTI FA channels.
The false positive map (Figure \ref{fig:ig:false-pos}) and the false negative
map (Figure \ref{fig:ig:false-neg}) should be interpreted differently since they
come from low confidence predictions; the probability of passing hovered on
either side of the pass/fail threshold. For example, in the false positive case,
the network is confused enough that it treats voxels that are outside of the
brain as informative as voxels in the major white matter bundles.

\begin{figure}[tbp]
    \centering
    \begin{subfigure}{\textwidth}
    \centering
    \includegraphics[width=0.9\linewidth]{deep-learning-qc/attribution-maps-true-pos.pdf}
    \caption{true positive}
    \label{fig:ig:true-pos}
    \end{subfigure}
    \begin{subfigure}{\textwidth}
    \centering
    \includegraphics[width=0.9\linewidth]{deep-learning-qc/attribution-maps-true-neg.pdf}
    \caption{true negative}
    \label{fig:ig:true-neg}
    \end{subfigure}
    \begin{subfigure}{\textwidth}
    \centering
    \includegraphics[width=0.9\linewidth]{deep-learning-qc/attribution-maps-false-pos.pdf}
    \caption{false positive}
    \label{fig:ig:false-pos}
    \end{subfigure}
    \begin{subfigure}{\textwidth}
    \centering
    \includegraphics[width=0.9\linewidth]{deep-learning-qc/attribution-maps-false-neg.pdf}
    \caption{false negative}
    \label{fig:ig:false-neg}
    \end{subfigure}
    \caption{%
        {\bf Integrated gradients attribution maps for the deep learning classifier}:
        Each column depicts a different channel of the input tensor: the B=0 DWI volume and
        the tensor FA images in the $x$, $y$, and $z$ directions. The first three columns
        show a horizontal slice while the last column shows a coronal slice. Blue voxels
        indicate positive attribution (i.e. evidence for passing the subject),
        while red voxels indicate negative attribution (i.e. evidence for QC
        failure). The underlying greyscale depicts the input channel. Each row
        depicts a representative subject from each confusion class:
        %
        \textbf{(a)} Attribution maps for a true positive prediction.  The model
        looks at the entire brain and focuses on known white matter bundles in
        the FA channels. In particular, it focuses on lateral bundles in the
        $x$ direction, anterior-posterior bundles in the $y$ direction, and
        superior-inferior bundles in the $z$ direction.
        %
        \textbf{(b)} Attribution maps for a true negative prediction. The model
        focuses primarily on the B=0 channel, suggesting that it ignores DTI FA
        when motion artifacts like banding are present.
        %
        \textbf{(c)} Attribution maps for a false positive prediction. Both the
        false positive and negative predictions were low confidence predictions.
        This is reinforced by the fact that the model views some voxels that are
        outside of the brain as just as informative as those in major white
        matter tracts.
        %
        \textbf{(d)} Attribution maps for a false negative prediction. The model
        fails to find long-range white matter tracts in the anterior-posterior
        and lateral directions. We also speculate that the model expects
        left-right symmetry in the FA channels and assigns negative attribution
        to asymmetrical features.
    }
    \label{fig:ig}
\end{figure}

\subsection*{Quality control improves inference}

To demonstrate the effect that quality control has on inference, we analyzed
tract profile data derived from HBN-POD2 data. Tract profiling
\cite{yeatman2012-rc,jones2005pasta,colby2012along,odonnell2009tract} is
a subset of tractometry \cite{jones2005pasta,bells2011tractometry}, which uses
the results of diffusion MRI tractography to quantify properties of the white
matter along specific pathways, or bundles. Tract-profiling retains the values
of diffusion metrics along the trajectory of each bundle, rather than computing
summary statistics at the level of each bundle. Previously, three of the co-authors
used tract profile data derived from HBN-POD2 to develop a grouped covariate penalized
regression model \cite{richie-halford2021multidimensional} and showed that
the tract profiles exhibit strong site differences. In Figure
\ref{fig:qc-profiles:md}, we plot mean diffusivity tract profiles grouped into
four QC bins along the length of twenty-four bundles:
While some bundles, such as the CGC and ILF, appear insensitive to
QC score, others, such as the UNC and orbital corpus callosum, exhibit
strong differences between QC bins. In most bundles, low QC scores
tend to flatten the profile.

\begin{figure}[tbp]
    \centering
    \includegraphics[width=\linewidth]{bundle-profiles/qc-bins-dki-md.pdf}
    \caption{%
        {\bf MD bundle profiles show large QC group differences}:
        MD profiles binned by QC score in twenty-four major while matter
        bundles. The left and right uncinate bundles are the most sensitive
        to QC score. Generally, QC score tends to flatten bundle profiles.
        Error bands represent 95\% confidence intervals. Bundle abbreviations
        for lateralized bundles contain a trailing ``L'' or ``R'' indicating the
        hemisphere. Bundle abbreviations:
        inferior fronto-occipital fasciculus (IFO),
        uncinate (UNC),
        thalamic radiation (ATR),
        corticospinal (CST),
        arcuate (ARC),
        superior longitudinal fasciculus (SLF).
        inferior longitudinal fasciculus (ILF),
        cingulum cingulate (CGC),
        orbital corpus callosum (Orbital),
        anterior frontal corpus callosum (AntFrontal),
        superior frontal corpus callosum (SupFrontal),
        motor corpus callosum (Motor),
        superior parietal corpus callosum (SupParietal),
        temporal corpus callosum (Temporal),
        post-parietal corpus callosum (PostParietal), and
        occipital corpus callosum (Occipital).
    }
    \label{fig:qc-profiles:md}
\end{figure}

\arhtodo[inline]{
    The tract profile plot needs improvement. Specifically, the axes text
    takes up a lot of room so it's hard to see the plots.
}

The effect of QC score on white matter bundle profiles suggests that researchers
using HBN-POD2 incorporate QC in their analyses, either by applying a QC cutoff
when selecting subjects or by explicitly added QC score to their inferential
models. Failure to do so may cause spurious associations or degrade predictive
performance. To demonstrate this, we select participant age as a representative
phenotypic benchmark \cite{cole2019brain,richie-halford2021multidimensional} and
observe the effect of varying QC cutoff on the predictive performance of an age
prediction model. Figure \ref{fig:age-prediction} shows the cross-validated
$R^2$ scores for age prediction model trained on subjects after exclusion using a
variable QC cutoff. The increase in model performance from imposing a QC cutoff
is intuitive: we know from \ref{fig:qc-profiles:md} that participants with
low QC scores have reduced MD, but MD also decreases as subjects mature
\cite{yeatman2014lifespan,richie-halford2021multidimensional}.
Eliminating subjects with low QC therefore removes subjects who may look
artificially older from the analysis, improving overall performance.  The most
noticeable improvement in performance comes after imposing the most modest
cutoff of $0.05$, suggesting that researchers will benefit from \emph{any} QC
screening. On the other hand, QC screening inherently introduces a tension
between the desire for high quality data and the desire for a large sample size.
In this case, after a QC cutoff of around $0.9$, the training set size is
reduced such that it degrades predictive performance. We do not expect the
sensitivity analysis of an age prediction model to generalize to other
analyses and recommend that researchers using HBN-POD2 conduct their own.

\begin{figure}[htbp]
    \centering
    \includegraphics[width=0.5\linewidth]{age-prediction/qc_sweep.pdf}
    \caption{%
        {\bf Imposing a QC cutoff improves age prediction}:
        Cross validated $R^2$ scores (left axis, blue dots) from an age
        prediction model increase after screening subjects by QC score.  We see
        the most dramatic increase in $R^2$ after imposing even the lowest
        cutoff of $0.05$. Thereafter, the $R^2$ scores trend upward until a
        cutoff of $\sim 0.95$, where the training set size (right axis, orange
        line) becomes too small to sustain model performance.
    }
    \label{fig:age-prediction}
\end{figure}

\section*{Discussion}

We present HBN-POD2, one of the largest youth diffusion imaging datasets with
derived measures currently available. It is openly available and complies with
the current draft of the BIDS diffusion derivative specification. It will grow
continuously as the HBN study acquires more data, eventually reaching its
\num{10000} subject goal. Here we summarize the contributions of this work,
starting from the potential impact of the dataset itself and ending with implications for
quality control of other large-scale neuroimaging datasets.

\subsection*{Standardized preprocessing reduces burden on community researchers}

The most immediate contribution of this work is a large analysis-ready dMRI
dataset, openly accessible to the public. In the past decade, projects such as
the HBN, Human Connectome Project (HCP) \cite{van-essen2013-oi}, UK Biobank
\cite{miller2016-mq}, ABCD \cite{jernigan2018-my}, and CamCAN
\cite{taylor2017-or,shafto2014-ld} have ushered a culture of data sharing in
open big-data neuroscience. The adoption and reuse of these datasets reduces or
eliminates the data collection burden on downstream researchers.  Some projects,
such as the HCP \cite{glasser2013-lo}, provide preprocessed diffusion
derivatives, further reducing researchers' burden and extending the benefits of
data-sharing from data collection to preprocessing and secondary analysis.
Following the example of the HCP, HBN-POD2 provides analysis-ready dMRI
derivatives. This avoids duplication of preprocessing effort while also ensuring
a minimum standard of data quality for HBN researchers. The data is amenable to
many different analyses, including tractometry \cite{yeatman2012-rc}, graph
theoretical analysis \cite{yeh2020-nu}, and combinations with functional data
for the same subjects.  The availability of standardized preprocessed diffusion
data will allow researchers to create and test hypotheses on the white matter
properties underlying behavior and disease, from reading and math acquisition to
childhood adversity and mental health. As such, this dataset will accelerate
discovery at the nexus of structural connectivity and neurodevelopmental and
learning disorders.

Importantly, the deep learning network architecture and parameters are also
provided as part of this work, allowing application of this network to future
releases of HBN data, and allowing other researchers to build upon and improve
upon this work. Moreover, undertaking this processing and QC effort required
construction and deployment of substantial informatics infrastructure, including
tools for cloud computing, web applications for expert annotation and for
community science rating and analysis software. All of these tools are provided
openly, so that this approach can be adopted widely in other projects and in
other scientific fields.

\subsection*{A preprocessing baseline for methods researchers}

While the primary audience of HBN-POD2 is researchers in neurodevelopment who
will use the dMRI derivatives in their studies, other researchers may use
HBN-POD2 to develop new preprocessing algorithms or quality control methods. In
this respect, HBN-POD2 follows Avesani et al.~\cite{avesani2019-ey}, who
recognized the diverse interests that different scientific communities have in
reusing neuroimaging data and coined the term \emph{data upcycling} to promote
multiple-use data sharing for purposes secondary to those of the original
project. In contrast with their open diffusion data derivatives, which provided
a small number of subjects preprocessed with many pipelines, HBN-POD2 contains
many subjects, all processed with a single state of the art pipeline,
\emph{QSIPrep}. For researchers developing new preprocessing algorithms,
HBN-POD2 provides a large, open baseline to which to compare their results.

\subsection*{A QC baseline for methods researchers}

Similarly, HBN-POD2 will serve as a baseline for quality control methods.
HBN-POD2 provides three separate QC scores alongside its large dataset of
pediatric neuroimaging diffusion derivatives. Including these scores will help
pediatric mental health researchers improve inferences derived from the imaging
data, as suggested by the improvements in age prediction shown in Figure
\ref{fig:age-prediction}. In addition, neuroimaging QC methods developers will
benefit from a large benchmark dataset with which to pilot new QC methods.

\rokemtodo[inline]{%
    Expound on the impact of the inclusion of QC scores from a data upcycling
    perspective.
}

\subsection*{Generalizing the hybrid QC approach to dMRI}

\arhtodo[inline]{%
    Add some text about the benefit of generalizing the hybrid approach to
    multisite and multi-modality data. Also need to discuss comparison with
    previous T1-weighted work. One thing to discuss there is that expert
    inter-rater reliability was *a lot* higher for T1-weighted. This is
    potentially a problem, but also highlights how important it is to have an
    automated, "objective" approach.
}

\arhtodo[inline]{%
    Mention that the XGB-q model performs really well. Not quite as good as the
    CNN-i model but it's more transparent and relies only on QSIPrep automated
    QC metrics. Maybe cite Rudin's paper \cite{rudin2019stop}. We release the
    CNN-i scores since they better reproduce the expert ratings but the XGB-q
    model is reproducible using the qc repo's docker commands and readers may
    want to use that if they want a more interpretable prediction.

    This is also a good place to cite Figure 9 of Tobe et
    al.~\cite{tobe2021longitudinal}, which shows that Euler number can reliably
    predict braindr scores in NKI-Rockland.
}

\rokemtodo[inline]{%
    Add some text about the benefit of generalizing the hybrid approach to
    multisite and multi-modality data.
}

\subsection*{Future work and open problems}

\rokemtodo[inline]{%
    Discuss future releases of HBN data. This seems to be an open problem
    affecting most multi-year projects, which data derivatives being released
    before study completion. The use of QSIPrep and containerization make it
    easy to run the same pipeline on new subjects and even assign QC scores to
    them. However, there is the question of what we will do when QSIPrep
    inevitable improves and releases a new version.
    Running again with improved methods will be burdensome, but here again we
    can spin that: remember that we provide a baseline against which future
    improvements will be compared.
}

\arhtodo[inline]{%
    Add a brief note about algorithmic impact. I remember some paper with
    recommendations for ML researchers to include an algorithmic impact
    statement in their discussion sections. I don't recall the exact paper
    though. But we should discuss potential impact of the CNN-i model. We've
    demonstrated that it is reliable for the HBN-POD2 dataset and will hint
    at its generalizability to other datasets. We did example based explanations
    using integrated gradients but are punting on a more thorough investivation
    of intepretability.

    But a reliable algorithm might still negatively influence expert's
    decisions, e.g. by spurring them to exclude populations deserving of study.
    Or by forging a less thorough research community, outsourcing decision
    making to algorithms without a clear understanding of the data.
    Caveat emptor.
}

\rokemtodo[inline]{%
    Speculate about generalization to other datasets. We should point out ABCD
    in particular since the CUNY and CBIC sites are supposed to be harmonized
    with ABCD. This discussion will hinge on the results of the remaining multi-site
    generalization experiments.
}

\section*{Methods}

To facilitate replicability, Jupyter notebooks \cite{kluyver2016jupyter} and
Dockerfiles \cite{merkel2014docker} necessary to reproduce the methods described
herein are provided in the HBN-POD2 GitHub repository at
\url{https://github.com/richford/hbn-pod2-qc}. The specific version of the
repository used in this study is also available at
\url{https://doi.org/our.awesome.doi}. The \texttt{make} or \texttt{make help}
commands will list the available commands and \texttt{make build} will build the
requisite Docker images to analyze HBN-POD2 QC data. In order to separate data
from analysis code \cite{Wilson2017-rj}, we provide intermediate data necessary
to analyze the QC results in an OSF \cite{Foster-MSLS2017-rl} project
\cite{hbn-pod2-osf}, which can be downloaded using the \texttt{make data}
command in the root of the HBN-POD2 GitHub repository. Most of the code in this
repository uses Pandas \cite{mckinney-proc-scipy-2010,reback2020pandas}, Numpy
\cite{harris2020array}, Matplotlib \cite{hunter2007matplotlib}, and Seaborn
\cite{waskom2021seaborn}.
\arhtodo[inline]{Zenodo the Github repo and add a citation here.}

Inputs for this study consisted of MRI data from the Healthy Brain Network
pediatric mental health study \cite{alexander2017-yc}, containing dMRI data from
\num{2747} subjects with ages 5-21. These data were measured using a
\qty{1.5}{\tesla} Siemens mobile scanner on Staten Island (SI) and three fixed
\qty{3}{\tesla} Siemens MRI scanners at sites in the New York area: Rutgers
University Brain Imaging Center (RUBIC), the CitiGroup Cornell Brain Imaging
Center (CBIC), and the City University of New York Advanced Science Research
Center (CUNY). Informed consent was obtained from each participant aged 18 or
older. For participants younger than 18, written consent was obtained from their
legal guardians and written assent was obtained from the participant. Voxel
resolution was \qtyproduct{1.8 x 1.8 x 1.8}{\mm} with \num{64} non-colinear
directions measured for each of $b=1000$ \unit{\second \per \mm^{2}} and
$b=2000$ \unit{\second \per \mm^{2}}.

\subsection*{BIDS curation}

We curated the imaging metadata for \num{2615} of the \num{2747} currently
available HBN subjects. Using dcm2bids and custom scripts, we conformed the data
to the Brain Imaging Data Structure (BIDS; \cite{gorgolewski2016-lh})
specification.  The BIDS-curated dataset is available on FCP-INDI and can be
accessed via AWS S3 at \url{s3://fcp-indi/data/Projects/HBN/BIDS_curated/}.

\mctodo[inline]{Add more BIDS curation information}

\subsection*{Preprocessing}

We performed dMRI preprocessing on \num{2136} subjects, using \emph{QSIPrep}
\cite{cieslak2021-iq} 0.12.1, which is based on \emph{Nipype} 1.5.1
\cite{nipype1,nipype2}, RRID:SCR\_002502. \emph{QSIPrep} a robust and scalable
pipeline to group, distortion correct, motion correct, denoise, coregister and
resample MRI scans.  In total, \num{417} subjects failed this preprocessing
step, largely due to missing dMRI files. In keeping with the BIDS specification,
the preprocessed dataset is available as a derivative dataset within the
BIDS-curated dataset and can be access on AWS S3 at
\url{s3://fcp-indi/data/Projects/HBN/BIDS_curated/derivatives/qsiprep/}.
\emph{QSIPrep} fosters reproducibility by automatically generating thorough
methods boilerplate for later use in scientific publications, which we use for
the remainder of this subsection to document each preprocessing step.

\begin{itemize}

\item {\it Anatomical data preprocessing}
The T1-weighted (T1w) image was corrected for intensity non-uniformity (INU)
using \texttt{N4BiasFieldCorrection} \cite{n4} (ANTs 2.3.1), and used as
T1w-reference throughout the workflow. The T1w-reference was then skull-stripped
using \texttt{antsBrainExtraction.sh} (ANTs 2.3.1), using OASIS as target
template. Spatial normalization to the ICBM 152 Nonlinear Asymmetrical template
version 2009c \cite{mni}, RRID:SCR\_008796 was performed through nonlinear
registration with \texttt{antsRegistration} \cite{ants}, ANTs 2.3.1,
RRID:SCR\_004757, using brain-extracted versions of both T1w volume and
template. Brain tissue segmentation of cerebrospinal fluid (CSF), white-matter
(WM) and gray-matter (GM) was performed on the brain-extracted T1w using
\texttt{FAST} \cite{fsl-fast}, FSL 6.0.3:b862cdd5, RRID:SCR\_002823.

\item {\it Diffusion data preprocessing}

Any images with a $b$-value less than \qty{100}{\second \per \mm^{2}} were treated
as a $b=0$ image. MP-PCA denoising as implemented in MRtrix3's
\texttt{dwidenoise}\cite{dwidenoise1} was applied with a 5-voxel window. After
MP-PCA, B1 field inhomogeneity was corrected using \texttt{dwibiascorrect} from
MRtrix3 with the N4 algorithm \cite{n4}.  After B1 bias correction, the mean
intensity of the DWI series was adjusted so all the mean intensity of the $b=0$
images matched across eachseparate DWI scanning sequence.

FSL (version 6.0.3:b862cdd5)'s eddy was used for head motion correction
and Eddy current correction \cite{anderssoneddy}. Eddy was configured
with a \(q\)-space smoothing factor of 10, a total of 5 iterations, and
1000 voxels used to estimate hyperparameters. A linear first level model
and a linear second level model were used to characterize Eddy
current-related spatial distortion. \(q\)-space coordinates were
forcefully assigned to shells. Field offset was attempted to be
separated from subject movement. Shells were aligned post-eddy. Eddy's
outlier replacement was run \cite{eddyrepol}. Data were grouped by
slice, only including values from slices determined to contain at least
\num{250} intracerebral voxels. Groups deviating by more than four standard
deviations from the prediction had their data replaced with imputed
values. Data was collected with reversed phase-encode blips, resulting
in pairs of images with distortions going in opposite directions. Here,
$b=0$ reference images with reversed phase encoding directions were used
along with an equal number of $b=0$ images extracted from the DWI scans.
From these pairs the susceptibility-induced off-resonance field was
estimated using a method similar to that described in \cite{topup}. The
fieldmaps were ultimately incorporated into the Eddy current and head
motion correction interpolation. Final interpolation was performed using
the \texttt{jac} method.

Several confounding time-series were calculated based on the
\emph{preprocessed DWI}: framewise displacement (FD) using the implementation
in \emph{Nipype} following the definitions by \cite{power-fd-dvars}. The DWI
time-series were resampled to ACPC, generating a \emph{preprocessed DWI run
in ACPC space}.

\end{itemize}

Many internal operations of \emph{QSIPrep} use \emph{Nilearn} 0.6.2
\cite{nilearn}, RRID:SCR\_001362 and \emph{DIPY} \cite{dipy}. For more details
of the pipeline, see
\href{https://qsiprep.readthedocs.io/en/latest/workflows.html}{the section
corresponding to workflows in \emph{QSIPrep}'s documentation}.

\subsection*{Cloud-based distributed preprocessing}

The containerization of \emph{QSIPrep} provided a consistent preprocessing
pipeline for each subject but the number of subjects made serial processing of
each particpant prohibitive on a single machine. We used \emph{cloudknot}, a
previously developed cloud-computing library \cite{cloudknot} to
parallelize the preprocessing over individual subjects on spot instances in the
Amazon Web Services Batch service. \emph{Cloudknot} takes as input a
user-defined Python function and creates the necessary AWS infrastructure to map
that function onto a range of inputs, in this case, the subject IDs. The Python
preprocessing function was a thin wrapper around \emph{QSIPrep}'s command line
interface and is provided in a jupyter notebook in the HBN-POD2 GitHub
repository in the ``notebooks'' directory.  Using \emph{cloudknot} and AWS Batch
Spot Instances, the preprocessing cost less than \textdollar1.00 per subject.

\subsection*{Expert quality control}

The expert QC ``gold standard'' subset was created by randomly selecting 200
subjects from the preprocessed dataset, sampled such that the proportional site
distribution in the gold standard subset matched that of the preprocessed
dataset.

We created a web application for expert quality control of preprocessed dMRI,
called \emph{dmriprep-viewer} \cite{richie-halford2021-viewer}. The viewer
ingests \emph{QSIPrep} outputs and generates a browser-based interface for
expert QC. It provides a study overview page displaying the distributions of
\emph{QSIPrep}'s automated QC metrics (described at
\url{https://qsiprep.readthedocs.io/en/latest/preprocessing.html#quality-control-data}).
Each datum on the study overview page is interactively linked to a subject-level
QC page that provides an interactive version of \emph{QSIPrep}'s visual reports
(described at
\url{https://qsiprep.readthedocs.io/en/latest/preprocessing.html#visual-reports}).
The viewer allows users to assign a rating of $-2$ (definitely fail), $-1$
(probably fail), $0$ (not sure), $1$ (probably pass), or $2$ (definitely pass) to a
subject. To standardize rater expectations before rating, expert raters watched
a tutorial video (which is available in the OSF project). They then rated each
subject and saved their scores and sent them to the lead author for compilation.

\comment[inline]{%
    The new viewer will be called NIRV, the neuroimaging report viewer and we'll
    write another paper about that. So lots of details about the implementation
    are witheld from this paper.
}

To compute the pairwise Cohen's $\kappa$ scores in \ref{fig:expert-qc:irr}, we
used the \emph{scikit-learn} \cite{scikit-learn} \texttt{cohen\_kappa\_score}
function with quadratic weights. To compute intraclass correlation, we used the
\emph{pingouin} statistical package \cite{vallat2018pingouin}
\texttt{intraclass\_corr} function. The expert rating analysis can be replicated
using the \texttt{make expert-qc} command in the HBN-POD2 GitHub repository.

\subsection*{Community scientist quality control}

The community science web application is based on the
\href{https://swipesforscience.org/}{SwipesForScience framework}, which
generates a web application for community science given an open repository of
images to be labelled and a configuration file. The source code for the
\emph{Fibr} web application is available at
\url{https://github.com/richford/fibr}.
After a brief tutorial, community scientists provided binary pass/fail ratings
based on the directionally-colorized fractional anisotropy from DTI of each
subject's preprocessed dMRI data. These images were generated using a \emph{DIPY}
\cite{dipy} \texttt{TensorModel} in a \emph{cloudknot}-enabled Jupyter notebook
that is available in the ``notebooks'' directory of the \emph{Fibr} GitHub
repository. \emph{Fibr} saves each community rating to its Google Firebase
backend, the contents of which have been archived to the HBN-POD2 OSF project.

The \emph{Fibr} ratings were then combined with automated \emph{QSIPrep} QC
metrics to train the gradient boosted trees models XGB, XGB-f, and XGB-q. These
models were implemented using the XGBoost library \cite{xgboost}. Using repeated
stratified K-fold cross-validation, with three splits and two repeats, we
evaluated the models' performance in predicting the gold standard ratings. In
each fold, the best model hyperparameters were chosen using the scikit-optimize
\cite{scikit-optimize} \texttt{BayesSearchCV} class. Saved model checkpoints for
each cross-validation split are available in the HBN-POD2 OSF project. Since
each split resulted in a difference XGB model and we required a single QC score
to train the deep learning model, we combined the models from each
cross-validation split using a voting classifier, computing a weighted averaged
of the predicted probability of passing from each model, weighted by its
out-of-sample ROC AUC. This was implemented using scikit-learn's
\texttt{VotingClassifier} class. Treating the voting classifier as another
``expert'' rater, we reassessed the pairwise Cohen's $\kappa$ and ICC scores as
in the expert QC subsection. The community ratings analysis can be replicated
using the \texttt{make community-qc} command in the HBN-POD2 GitHub repository.

\subsection*{Deep learning to predict quality control}

\begin{figure}[tbp]
    \begin{subfigure}[t]{0.6\textwidth}
    \centering
    \includegraphics[width=\linewidth]{deep-learning-qc/model.pdf}
    \caption{Slicing and combining the input channels}
    \label{fig:dl-architecture:complete}
    \end{subfigure}
    \begin{subfigure}[t]{0.4\textwidth}
    \centering
    \includegraphics[width=\linewidth]{deep-learning-qc/image_model.pdf}
    \caption{CNN architecture}
    \label{fig:dl-architecture:cnn}
    \end{subfigure}
    \caption{%
        {\bf Deep learning model architecture}:
        \textbf{(a)} The CNN-i+q model accepts multichannel input that combined
        four imaging channels with a fifth channel containing 31 \emph{QSIPrep}
        automated QC metrics.  The imaging channels are separated from the QC
        channel using \texttt{Lambda} layers. The imaging channels are passed
        through a CNN \textbf(b), the output of which is concatenated with the
        QC metrics, batch normalized and passed through two fully-connected (FC)
        layers, with ReLu activation functions and with 512 and 128 units
        respectively. Each FC layer is followed by a dropout layer which drops
        40\% of the input units. The final layer contains a single unit with a
        sigmoid activation function and outputs the probability of passing QC.
        %
        \textbf{(b)} The CNN portion of the model passes the imaging input
        through four convolutional blocks. Each block consists of a 3D
        convolutional layer with a kernel size of 3 and a ReLu activation, a 3D
        max pooling layer with a pool size of 2, and a batch normalization layer
        with Tensorflow's default parameters.  The number of filters in the
        convolutional layers in each block are 64, 64, 128, and 256 respecively.
        The output of the final block is passed through a 3D global average
        pooling layer with Tensorflow's default parameters.
    }
    \label{fig:dl-architecture}
\end{figure}

The voting classifier's predictions were then used as targets to train a deep
learning classifier to predict QC scores based on each subject's preprocessed
dMRI data. We trained two different model architectures
\begin{enumerate*}[%
    label=(\roman*),%
    before=\unskip{: },%
    itemjoin={{, }},%
    itemjoin*={{ and }}]
    \item CNN-i, which took only preprocessed dMRI images as input
    \item CNN-i+q, whose input also included \emph{QSIPrep}'s automated QC metrics.
\end{enumerate*}
Both models were implemented in Tensorflow 2 \cite{tensorflow} using the Keras
module \cite{keras}. The image processing part of the model architecture was
identical for both models: a modification of an existing 3D CNN
\cite{zunair2020-bs} previously applied to the assess tuberculosis severity
\cite{dicente2019clef}. It accepts a 3D volume as input with four channels
\begin{enumerate*}[%
    label=(\roman*),%
    before=\unskip{: },%
    itemjoin={{, }},%
    itemjoin*={{ and }}]
    \item the B=0 reference volume
    \item tensor FA in the $x$-direction
    \item tensor FA in the $y$-direction
    \item tensor FA in the $z$-direction.
\end{enumerate*}
The \emph{QSIPrep}'s automated QC metrics were included as an additional fifth
channel. The CNN-i+q model architecture is summarized in Figure
\ref{fig:dl-architecture}. Upon input, the CNN-i+q model extracts the imaging
channels and passes them through the CNN architecture. The remaining QC metric
channel is flattened and passed ``around'' the CNN architecture and concatenated
with the output of the convolutional layers. This concatenated output is then
passed through a fully-connected layer to produce a single output, the
probability of passing QC. This architecture has 1,438,783 trainable parameters.

We used \emph{DIPY} \cite{dipy} and \emph{cloudknot} \cite{cloudknot} to
generate these multichannel volumes for each subject and save them as NIfTI-1
files \cite{nifti}. These NIfTI files were then converted to the Tensorflow
TFRecord format using the \emph{Nobrainer} deep learning framework
\cite{nobrainer}.
\arhtodo[inline]{%
    Before submission, ask the nobrainer folks to cut a new release that
    contains Adam's contributions so that we are referencing the correct
    version. This cites a previous version that will not accept multichannel
    volumes.
}
The Jupyter notebooks used to create these NIfTI and TFRecord files are
available in the ``notebooks'' directory of the \emph{Fibr} GitHub
repository.

We trained each model using the Google Cloud AI Platform Training service and
the HBN-POD2 GitHub repository contains Docker services to launch training (with
\texttt{make dl-train}) and prediction (with \texttt{make dl-predict}) jobs on
Google Cloud, provided the user has provided the appropriate credentials in an
environment file and placed the TFRecord files on Google Cloud Storage. To
estimate the variability in model training, we trained ten separate models using
different training and validation splits of the data. The gold standard dataset
was not included in any of these splits and was reserved for reporting final
model performance. Models were optimized for binary crossentropy loss using the
Adam optimizer \cite{kingma2017adam} with an initial learning rate of 0.0001. We
reduced the learning rate by a factor of 0.5 when the validation loss plateaued
for more than two epochs. We also stopped training when the validation loss
failed to improve by more than 0.001 for twenty consecutive epochs.  These two
adjustments were made using the \texttt{ReduceLROnPlateau} and
\texttt{EarlyStopping} callbacks in Tensorflow 2 \cite{tensorflow} respectively.
The training and validation loss curves for both the CNN-i and CNN-i+q models
are depicted in Figure \ref{fig:dl-loss}. While the CNN-i+q model achieved
better validation loss, it did not outperform the CNN-i model on the held out
gold standard dataset.

\begin{figure}[tbp]
    \begin{subfigure}[t]{0.5\textwidth}
    \centering
    \includegraphics[width=0.9\linewidth]{deep-learning-qc/dl_learning_curve_with_qc.pdf}
    \caption{CNN-i+q}
    \label{fig:dl-loss:both}
    \end{subfigure}
    \begin{subfigure}[t]{0.5\textwidth}
    \centering
    \includegraphics[width=0.9\linewidth]{deep-learning-qc/dl_learning_curve_without_qc.pdf}
    \caption{CNN-i}
    \label{fig:dl-loss:imaging}
    \end{subfigure}
    \caption{%
        {\bf Deep learning model loss curves}:
        The binary cross-entropy loss (top), accuracy (middle), and ROC AUC
        (bottom) for the CNN-i model (left) and the CNN-i+q model (right). Model
        performance typically plateaued after twenty epochs but was allowed
        continue until meeting the early stopping criterion.
    }
    \label{fig:dl-loss}
\end{figure}

To generate the attribution maps, we followed Tensorflow's integrated gradients
tutorial \cite{integrated-gradients-tutorial} with a black baseline image and
128 steps in the Reimann sum approximation of the integral (i.e.
\texttt{m\_steps = 128}). In the HBN-POD2 GitHub repository, we provide a Docker
service to compute integrated gradient attribution maps on Google Cloud, which
can be invoked using the \texttt{make dl-integrated-gradients} command.

\subsection*{QC bundle profiles}

To generate bundle profiles, reconstruction was performed using the
\emph{QSIprep} 0.12.1 preconfigured reconstruction workflow
\texttt{mrtrix\_multishell\_msmt}, modified to generate two million streamlines
rather than the default ten million.  Multi-tissue fiber response functions were
estimated using the dhollander algorithm. FODs were estimated via constrained
spherical deconvolution (CSD, \cite{originalcsd, tournier2008csd}) using an
unsupervised multi-tissue method \cite{dhollander2019response,
dhollander2016unsupervised}. Reconstruction was done using MRtrix3
\cite{mrtrix3}. FODs were intensity-normalized using mtnormalize
\cite{mtnormalize}.

These tractograms were then used as input to the Python Automated Fiber
Quantification toolbox (pyAFQ) \cite{kruper2021evaluating}.  Twenty-four major
tracts, which are enumerated in \ref{fig:qc-profiles:md}, were identified using
multiple criteria: streamlines are selected as candidates for inclusion in a
bundle of streamlines that represents a tract if they pass through known
inclusion ROIs and do not pass through exclusion ROIs \cite{Wakana2007-nw}. In
addition, a probabilistic atlas is used to exclude streamlines which are
unlikely to be part of a tract \cite{Hua2008-di}. Each streamline is resampled
to 100 nodes and the robust mean at each location is calculated by estimating
the 3D covariance of the location of each node and excluding streamlines that
are more than 5 standard deviations from the mean location in any node. Finally,
a bundle profile of tissue properties in each bundle was created by
interpolating the value of MRI maps of these tissue properties to the location
of the nodes of the resampled streamlines designated to each bundle. In each of
100 nodes, the values are summed across streamlines, weighting the contribution
of each streamline by the inverse of the mahalanobis distance of the node from
the average of that node across streamlines. This means that streamlines that
are more representative of the tract contribute more to the bundle profile,
relative to streamlines that are on the edge of the tract.

These processes create bundle profiles, in which diffusion measures are
quantified and averaged along twenty-four major fiber tracts. We retain only the
mean diffusivity (MD) and the fractional anisotropy (FA) from a diffusion
kurtosis imagin (DKI) model \cite{jensen2005-ta}, and impute missing bundles
using median imputation as implemented by \emph{scikit-learn}'s
\texttt{SimpleImputer} class. Because the HBN-POD2 bundle
profiles exhibit strong site effects \cite{richie-halford2021multidimensional},
we used the ComBat harmonization method to robustly adjust for site effects in
the tract profiles. Initially designed to correct for site effects in gene
expression studies \cite{Johnson2007-kl}, ComBat employs a parametric empirical
Bayes approach to adjust for batch effects and has since been applied to
multi-site cortical thickness measurements \cite{fortin2018-hk}, multi-site DTI
studies \cite{fortin2017-be}, and functional MRI data in the Adolescent Brain
Cognitive Development Study (ABCD) \cite{nielson2018detecting}. We rely on the
\emph{neurocombat\_sklearn} library \cite{neurocombat-sklearn}, to apply ComBat in
before plotting bundle profiles in Figure \ref{fig:qc-profiles:md} using
plotting functions from the AFQ-Insight package
\cite{richie-halford2019insight}.  The bundle profile analysis can be replicated
using the \texttt{make bundle-profiles} command in the HBN-POD2 GitHub
repository.

\subsection*{Brain age prediction}

We evaluated the effect of varying the QC cutoff on model performance by
observing cross-validated $R^2$ values of gradient boosted trees models
implemented using XGBoost. The input feature space for each model consisted of
4800 features per subject, comprising 100 nodes for each of MD and FA in the
twenty-four major tracts. We imputed missing bundles and harmonized the
different scanning sites as above. The XGBoost models' hyperparameters were
hand-tuned to values that have been performant in the authors' previous
experience. We log-transformed each participant's age before prediction using
\emph{scikit-learn}'s \texttt{TransformedTargetRegressor} class. For each value
of the QC cutoff between 0 and 0.95, in steps of 0.05, we computed the
cross-validated $R^2$ values using \emph{scikit-learn}'s
\texttt{cross\_val\_score} function with repeated K-fold cross-validation using
five folds and five repeats. Note that the practice of whole dataset imputation
before cross-validation permits the possibility of dataset leakage between train
and test splits. Similarly, sequential hand-tuning of hyperparameters can lead
to unexpected overfitting \cite{hosseini2020itried}. A detailed study of aging
and the white matter would need to correct for these transgressions.  For this
study we deemed the risk of overfitting acceptable for the less fraught
demonstration of the effect of varying the QC cutoff.

\bibliography{hbn-pod2}

\section*{Acknowledgements}

We would like to thank Anisha Keshavan for useful discussions of community
science and web-based quality control. This manuscript was prepared using a
limited access dataset obtained from the Child Mind Institute Biobank, The
Healthy Brain Network dataset. This manuscript reflects the views of the authors
and does not necessarily reflect the opinions or views of the Child Mind
Institute.
\alltodo[inline]{Add grant numbers.}

\section*{Author contributions statement}

The first author named is the corresponding author. The last two authors named
share senior authorship. The first two authors named share lead authorship. The
remaining authors are listed in alphabetical order, with the exception of
the Fibr Community Science Consortium, whose members provided community science
QC ratings.

\alltodo[inline]{Please add your initials here as appropriate}

We describe contributions to the paper using the CRediT taxonomy \cite{brand2015-vd,allen2014-oc}:
\begin{itemize}
    \item Conceptualization: A.R-H., A.R., T.S., and M.C.;
    \item Methodology: A.R-H. and A.R.;
    \item Software: A.R-H. and M.C.;
    \item Validation: ;
    \item Formal Analysis: A.R-H. and M.C.;
    \item Investigation: A.R-H. and M.C.;
    \item Resources: M.M.;
    \item Data curation: M.C. and L.A.;
    \item Writing – Original Draft: A.R-H. and A.R.;
    \item Writing – Review \& Editing: ;
    \item Visualization: A.R-H.;
    \item Supervision: A.R. and T.S.;
    \item Project Administration: A.R-H. and A.R.;
    \item Funding Acquisition: A.R. and T.S.
\end{itemize}

\section*{Additional information}

To include, in this order: \textbf{Accession codes} (where applicable);
\textbf{Competing interests} (mandatory statement).

The corresponding author is responsible for submitting a
\href{http://www.nature.com/srep/policies/index.html#competing}{competing
interests statement} on behalf of all authors of the paper. This statement must
be included in the submitted article file.

\arhtodo[inline]{Add accession codes and competing interests statement.}

% \begin{figure}[ht]
% \centering
% \includegraphics[width=\linewidth]{stream}
% \caption{Legend (350 words max). Example legend text.}
% \label{fig:stream}
% \end{figure}

% \begin{table}[ht]
% \centering
% \begin{tabular}{|l|l|l|}
% \hline
% Condition & n & p \\
% \hline
% A & 5 & 0.1 \\
% \hline
% B & 10 & 0.01 \\
% \hline
% \end{tabular}
% \caption{\label{tab:example}Legend (350 words max). Example legend text.}
% \end{table}

% Figures and tables can be referenced in LaTeX using the ref command, e.g. Figure \ref{fig:stream} and Table \ref{tab:example}.

\end{document}
